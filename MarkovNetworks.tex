A Markov random field (often abbreviated as MRF), Markov network or undirected graphical model is a 
set of random variables having a Markov property described by an undirected graph. 

A Markov random field is similar to a Bayesian network in its representation of dependencies; the differences 
being that Bayesian networks are directed and acyclic, whereas Markov networks are undirected and may be cyclic. 

Thus, a Markov network can represent certain dependencies that a Bayesian network cannot (such as cyclic dependencies); 
on the other hand, it can't represent certain dependencies that a Bayesian network can (such as induced dependencies).


When the probability distribution is strictly positive, it is also referred to as a Gibbs random field, because, 
according to the Hammersley–Clifford theorem, it can then be represented by a Gibbs measure. 

The prototypical Markov random field is the Ising model; indeed, the Markov random field was introduced as the 
general setting for the Ising model.[1] In the domain of artificial intelligence, a Markov random field is used to 
model various low- to mid-level tasks in image processing and computer vision.

For example, MRFs are used for image restoration, image completion, segmentation, image registration, 
texture synthesis, super-resolution, stereo matching and information retrieval.
