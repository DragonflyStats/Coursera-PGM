
\section*{Belief propagation}

Belief propagation, also known as Sum-product message passing is a message passing algorithm for performing inference on graphical models, such as Bayesian networks and Markov random fields.

It calculates the marginal distribution for each unobserved node, conditional on any observed nodes. Belief propagation is commonly used in artificial intelligence and information theory and has demonstrated empirical success in numerous applications including low-density parity-check codes, turbo codes, free energy approximation, and satisfiability.

 

The algorithm was first proposed by Judea Pearl in 1982, who formulated this algorithm on trees, and was later extended to polytrees. It has since been shown to be a useful approximate algorithm on general graphs.


Gaussian belief propagation is a variant of the belief propagation algorithm when the underlying distributions are Gaussian.

%=================================================%
\end{document}

